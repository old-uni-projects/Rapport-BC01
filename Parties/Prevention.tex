La prévention des risques naturels et technologiques englobe la totalité des dispositifs mis en œuvre afin de minimiser les conséquences d'un évènement d'origine naturelle ou humaine. La \textbf{prévention} s'inscrit totalement dans une approche de développement durable étant donné qu'elle vise à maintenir les conditions  sociales, économiques et environnementales sur le long terme. A l'inversion des actions de \textbf{réparation} qui cherchent à rétablir ou surpasser les conditions \textit{ante-crises}.

\section{L'étude des risques, une étape essentielle}

Afin de nous préserver d'une catastrophe, il est nécessaire d'en étudier ses causes. Ces études nous permettrons d'agir en amont afin de réduire notre \textbf{vulnérabilité} (cf. \nameref{ch:Lexique}) à ces derniers. 

Et même s'il est impossible d'annuler la probabilité d'un malheureux évènement, il est cependant envisageable de définir des \textbf{\textit{scénarios de réactions et divers plans d'actions}} qui viendront apporter une réponse cohérente, efficace, réfléchie \textit{(et testée si possible)} en amont.

Vient alors le besoin de centraliser les informations recueillies, de les capitaliser et enfin de les analyser. Tout ce processus dans l'objectif sous-jacent de pouvoir \textbf{prévoir} et \textbf{anticiper le risque}. Si possible, il serait également souhaitable d'en déduire des \textbf{lignes de conduite} pour réduire la probabilité d'occurrence de la catastrophe et des \textbf{axes d'amélioration} pour limiter les effets d'une éventuelle crise.

C'est dans cet objectif que de nombreux acteurs publiques, comme Météo France, recueillent et centralisent des données sur notre environnement (\textit{exemple: météorologiques, sismiques, \ldots}). 

\subsection{Anticipation et Prévoir ?=? Observation et Vigilance}

Suites aux nombreuses études menées au niveau international sur les catastrophes naturelles, les scientifiques ont pu établir, pour de nombreux phénomènes naturels, un ensemble de variables qui influent sur l'occurrence de chacun de ces derniers.\\
Tout système ayant pour but la détection des facteurs augmentant la probabilité de réalisation d'un risque majeur se doit d'observer en continue les variables qui influent sur ces derniers. En cas de détection positive, il appartient aux autorités compétentes de déclencher les plans d'actions prévus et d'alerter les populations concernées. L'usage des médias au sens large peut être requis afin d'informer   de manière totale ces dernières. L'usage de sirènes, hauts parleurs, messages sms et autres peuvent également être de puissants vecteurs de communication.

Cependant certains phénomènes, comme une crue rapide ou un effondrement de terrains, sont difficiles voir impossibles à prévoir avec précision. Il est donc complexe d'y appliquer un quelconque système d'alerte, où le cas échéant une procédure d'évacuation.

\subsection{La prise en compte des risques lors de l'aménagement.}

<< Depuis quelques années, le MEDD (2003) tente avec des collectivités volontaires de construire un projet local de prévention des risques qui leur permette de poursuivre le développement de leurs territoires >> [Catherine CARRÉ]~\cite{Inondations}. Ce changement de paradigme vise à minimiser les dégâts provoqués par une éventuelle catastrophe naturelle ou technologique et incite à repenser entièrement l'aménagement des territoires. 

C'est dans cette logique que des politiques territoriales limitant l'utilisation des zones à risques ont été mises en places. On peut également remarquer la présence de projets visant à réduire la vulnérabilité de zones déjà urbanisées. On pourra citer pour exemple les << actes de dépoldérisation à visée protectrice >> [Lydie GOELDNER-GIANELLA]~\cite{Polder}.

Les \textbf{P}lans de \textbf{P}révention des \textbf{R}isques naturels prévisibles (PPR), institués par la loi BARNIER (1995)~\cite{PPRNat} et les PPR Technologiques institués par la loi du 30 Juillet 2003~\cite{PPRTech} se positionnent dans la droite lignée de ces actions. Chacun d'eux se présente comme le référent central sur la question de la prévention des risques naturels et technologiques. Leur but commun est de permettre un développement plus sûr sur l'ensemble du territoire par le contrôle de ce dernier sur les zones jugées à risque. Ces derniers sont décidés sous l'impulsion du Préfet et sont réalisés par les services de l'état.

Dès sa réalisation, l'aménagement local ne peux se faire qu'avec la prise en compte de ces documents. Cela implique une limitation du développement en fonction de l'aléa (cf. \nameref{ch:Lexique}). La limitation pouvant aller jusqu'à l'interdiction totale de tout futur développement dans les zones les plus exposées.

\subsection{Le Retour sur expérience.}

Les accidents technologiques sont sujets, depuis un certains temps déjà, à de nombreuses analyses \textit{a posteriori}. Nous pourrions par exemple citer le travail d'Anni BORZEIX et al.~\cite{borzeix}, dans son travail d'analyse de la crise AZF. Mais également le travail de Nicolas DECHY et al.~\cite{dechy} qui cherche à tirer des "leçons" de cette catastrophe tout en constatant les limites de la directive européenne \href{http://www.ineris.fr/aida/consultation_document/1097/}{2012/18/UE (anciennement 82/501/CEE) - SEVESO}.

Les catastrophes naturelles sont également soumises à de nombreuses analyses. Toutes ces études visent à améliorer l'ensemble des processus gravitant autour de la situation de crise (remboursement par les assurances, plans de prévention, plans de secours, \ldots) voir à identifier de potentiels points d'amélioration sur le plan législatif.

\subsection{La prévention en amont.}

Un des moyens de réduire les conséquences d'une catastrophe est de communiquer auprès de la population pour donner les comportements à adopter en cas réalisation de l'aléa. 

Je me permettrai ici une petite digression pour fait d'exemple :

\textit{Je me rappelle encore de ce tristement célèbre  21 Septembre 2001\ldots  J'étais comme tous les jours de la semaine dans une école élémentaire de la banlieue Toulousaine, quand nous avons entendu la détonation. Au vue de la violence de cette dernière le personnel enseignant a immédiatement appliqué les consignes de confinement préconisées par les autorités dans de pareil cas. Ces actes, bien que de bon sens, n'ont pas été improvisés. Elles sont le fruit d'une communication effectuée en amont de la date fatidique. Bien que l'efficacité d'une telle mesure soit discutable puisque les personnes au plus proche de la catastrophe ont vu la totalité de leurs vitres soufflées par l'explosion et donc rendu tout confinement absurde, nous pouvons constater que le message des autorités avait bien correctement circulé. Les instructions suivantes furent communiquées très rapidement par radio et télévision.}

Dans d'autres cas ces \textit{gestes qui sauvent} auraient pu sauver la vie de nombreuses personnes de l'agglomération toulousaine. Par chance, les retombées chimiques furent, a priori, négligeables et imperceptibles sur la santé des populations. On comprend donc aisément l'importance d'une communication préventive unilatérale et qui doit être supporté par le système éducatif afin de rendre automatique et instinctif ces fameux gestes.

\clearpage
