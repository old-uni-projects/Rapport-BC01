La prévention des risques naturels et technologiques englobe la totalité des dispositifs mis en œuvre afin de minimiser les conséquences d'un évènement d'origine naturelle ou humaine. La \textbf{prévention} s'inscrit totalement dans une approche de développement durable étant donné qu'elle vise à maintenir les conditions  sociales, économiques et environnementales sur le long terme. A l'inversion des actions de \textbf{réparation} qui cherchent à rétablir ou surpasser les conditions \textit{ante-crises}.

\section{L'étude des risques, une étape essentielle}

Afin de nous préserver d'une catastrophe, il est nécessaire d'en étudier les causes. Ces études nous permettrons d'agir en amont afin de réduire notre \textbf{vulnérabilité} à ces derniers. 

Et même s'il est impossible d'annuler la probabilité d'un malheureux évènement, il est cependant envisageable de définir des \textbf{\textit{scénarios de réactions et divers plans d'actions}} qui viendront apporter une réponse cohérente, efficace, réfléchie \textit{(et testée si possible)} en amont.

Vient alors le besoin de centraliser les informations recueillies, de les capitaliser et de les analyser. Dans l'objectif évident de pouvoir \textbf{prévoir} et \textbf{anticiper le risque}. Si possible, il serait également souhaitable de tirer des \textbf{lignes de conduite} pour prévenir la catastrophe et des \textbf{plans d'actions} pour limiter les effets d'une éventuelle crise.

C'est dans cet objectif que de nombreux acteurs publiques, \textit{comme Météo France}, recueillent et centralisent des données sur notre environnement (\textit{exemple: météorologiques, sismiques, \ldots}). 

\subsection{Anticipation et Prévoir = Observation et Vigilance ?}

Suites aux nombreuses études menées au niveau international sur les catastrophes naturels, les scientifiques ont pu établir, pour de nombreux phénomènes naturels, un ensemble de variables qui influent sur l'occurrence de chacun de ces derniers%h.
Tout système de détection de risque se doit d'observer en continue les 


\clearpage
