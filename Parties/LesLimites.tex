\section{Avant-Propos}

Il est important de souligner que ce chapitre relève plus d'une \textbf{réflexion personnelle argumentée} que d'un fait avéré et prouvé mathématiquement. 

Il est tout à fait possible que mon argumentaire ne soit pas exact ou suffisamment argumenté en de nombreux points. Cependant ce travail se positionne comme un\textbf{ travail d'écriture} bibliographique de fin d'études de cycle ingénieur. Il me semble donc primordial de ne pas rendre un travail passif de pure analyse et capitalisation de données mais bien d'inscrire celui-ci dans une logique ingénieur par \textbf{appropriation} du sujet et \textbf{remise en question} des postulats qui me semblent que partiellement exacts.

\section{"Concertation" ou le nouveau "\textit{Must-Have}".}

Soulignée comme la \textbf{"Solution"} avec un grand S, la concertation est mise en avant comme l'étape obligatoire qui résoudra dans un futur plus ou moins proche nos nombreux problèmes dans la gestion des risques majeurs naturels et technologiques.

Voici quelques citations extraites du corpus BC01 illustrant ce propos :

\begin{chapquote}
{Catherine Carré et al\cite{Inondations}, \textit{page 7}}
L’opposition très forte des élus aux PPR\textbf{ est venue en partie de la dénonciation de l’absence de concertation} avec les services de l’État et du manque d’information de la société civile (Pottier et alii, 2004).
\end{chapquote}

\begin{chapquote}
{Catherine Carré et al\cite{Inondations}, \textit{page 11}}
Vers la reconnaissance de la concertation comme élément indispensable de l’adaptation des territoires aux processus de gestion des risques.
\end{chapquote}

\begin{chapquote}
{Nicolas Dechy et al\cite{dechy}, \textit{page 8}}
Le Parlement européen a [...] demandé un changement de logique en raison de « l’impossible risque zéro » vers une logique « d’éloignement du risque ». Il a incité les Etats membres à revoir leurs politiques d’urbanisation autour des sites à risques \textbf{ainsi que les procédures de concertation avec le public}.
\end{chapquote}

\begin{chapquote}
{Laurent MICHEL\cite{LaurentMICHEL}, \textit{page 1}}
Après cette catastrophe, plusieurs enquêtes et de nombreux témoignages ont conduit le gouvernement à proposer une loi pour renforcer la politique de prévention des risques accidentels dans les installations classées, sur quatre points qui en sont les piliers [...] \textbf{3. La concertation et la participation du public et des salariés} [...]
\end{chapquote}

\begin{chapquote}
{Myriam MERAD\cite{merad}, \textit{Titre de l'article}}
« LA CONCERTATION : UNE NOUVELLE DIMENSION » 
\end{chapquote}

\section{Pourquoi je n'y crois pas, du moins à court terme ?}

\subsubsection{Le parallèle Européen}

Afin d'illustrer mon propos je m'appuierai sur une situation actuelle que je pense similaire: L'exemple de la Démocratie Européenne. 

L'Europe a mis en place un système de concertation, où nous, citoyen européen, pouvons faire remonter un problème via un outils qui s'apparente aux pétitions appelé\textit{ Initiative citoyenne européenne} (\url{http://ec.europa.eu/citizens-initiative/public/welcome?lg=fr}). Si vous obtenez suffisamment de signatures (1 Million provenant d'au moins 7 Pays membres), votre requête sera entendue au parlement européen.

En théorie c'est fantastique, seulement quelles sont les limites pratiques d'un tel système (liste non exhaustive):\\
\begin{itemize}
	\item Qui peut se permettre de sacrifier des mois entiers pour obtenir les signatures nécessaires en plus de son travail régulier et de la vie de famille ?\\
	\item Qui a les connaissances linguistiques nécessaires pour s'adresser à sept pays différents ?\\
	\item Qui est même au courant de cette possibilité d'action ? \\
	\item Qui connait même le rôle de la Commission Européenne, qui sont nos élus ? Quels sont les sujets récents étudiés ?
\end{itemize}
 \vspace{5mm}

Ce n'est pas un malheureux hasard si le taux d'abstention lors de chaque élection européenne dépasse à chaque fois les 50\% (2014: 56\%, 2009: 59.36\%). Le système politique européen est obscure, difficile d'accès, et peu relayé par les médias.\\
Nous ne savons pas ce qui s'y fait et de quoi on y discute. A moins bien sûr de faire une démarche volontaire de recherche. Comportement que nous pouvons admettre, je pense, minoritaire.

\subsubsection{L'Europe, un modèle d'echec de concertation}

Aujourd'hui, je considère l'Europe comme un \textbf{modèle d'échec} en terme de relations publiques, d'échanges avec les citoyens et de concertation. Les rares initiatives françaises de \textbf{référendum} se sont soldées par l'ignorance de l'avis du peuple (\textit{Non à la constitution Européenne}) et un passage en force suite à de légères modifications (Référendum du 29 mai 2005 en France).

Il me semble donc peu ambitieux de conclure que volonté politique, enjeux économiques, complexité du sujet mènent tous ensembles, non seulement au désintérêt des populations mais également à la désinformation.
Il est tellement plus facile d'influencer quelqu'un qui ne comprend qu'en apparence un sujet.

De plus, je ne crois pas qu'une population donnée de personnes concernées par un même risque ou problème soit dans la capacité d'apprécier le travail d'experts et surtout de le remettre en question si nécessaire. Et dans ce cas on peut se demander quelle est l'intérêt d'obtenir un "Oui, je suis d'accord" si la personne n'est pas elle-même en capacité de fournir un regard critique sur cette situation.

\subsubsection{Quels sont les verrous d'une concertation efficace ?}

Je trouve personnellement qu'il n'est rien de plus hypocrite que de rendre une ressource accessible en moyens et seulement profitable à un comité réduit tout en ventant les mérites et les avancées que permettent l'accessibilité à tous.

Il suffit de se rendre sur n'importe quel site ministériel, l'information est complexe à trouver si l'on ne sait pas où exactement chercher. Les sites sont très austères et ne sont en rien dédiés à la majorité.\\

Voici une liste des critères qui sont essentiels, à mes yeux, pour la réussite d'un projet de concertation:\\
\begin{itemize}
	\item Une source d'information \textbf{officielle et centrale}. Il devrait être impensable de \textit{"jouer au ping pong"} entre les différents acteurs et services pour obtenir un quelconque renseignement.\\
	
	\item Un réel travail de \textbf{vulgarisation technico-scientifique} doit être mené à tous les niveaux sur tous les supports de communication. Chaque personne concernée doit être capable de comprendre les enjeux et les leviers d'action qui s'offrent à lui.\\
	
	\item Toujours dans un soucis d'accessibilité, il faut éviter de noyer les comptes rendus de séances et autres documents officiels sous des montagnes de références législatives et officielles. Si ces rapports sont destinés à être partagés et donc à être lu, alors qu'ils soient \textbf{courts, clairs, simples et concis}.\\
	
	\item La communication doit-être totale et non en "Best Effort" \textit{(Anglais: Au mieux possible sans garantie)} comme pourraient dire les Américains. Pour toucher un large publique aucun support ne doit être négligé.\\
	
	\item L'architecture des site-webs se doit d'être simple et attrayante. \textbf{Officiel et sérieux ne doit pas rimer avec austère et complexe}.\\
	
	\item Pour que \textbf{concertation} soit le mot adapté à la situation, il faut que les populations soient intégré dès le lancement de chaque projet et non à la fin. Le cas contraire nous sommes dans une situation d'information qui cherche à donner l'illusion d'un semblant de contrôle.\\
	
	\item Définir clairement les objectifs d'un groupe de travail.  Je ne crois pas que c'est en s'asseyant tous autour d'une table sans objectifs et ordres du jour que les problématiques d'aujourd'hui se règleront. Chaque groupe de travail pouvant être focalisé sur des problématiques différentes autour d'un même sujet ou même travailler sur des approches différentes.\\
\end{itemize}
\vspace{5mm}


Par soucis d'honnêteté, je tiens cependant à souligner l'excellent travail, que j'ai eu la bonne surprise de découvrir, sur le site internet du gouvernement concernant les risques: \url{http://www.risques.gouv.fr/}. Cela montre une réelle volonté et motivation politique, essentielle à toute entreprise de ce type.
Cependant, cela ne constitue qu'un seul des maillons de cette chaine,  et il y a encore de très nombreux manquant pour mener une politique de concertation efficacement.\\

C'est un objectif qu'il faut évidement poursuivre mais qui n'est en rien atteignable dans un futur proche.

\subsubsection{Quels seraient alors les recours citoyens envisageable pour demain ?}

Si la concertation est l'objectif de la semaine prochaine, alors nous sommes en droit de nous demander quel sera le recours citoyen de demain ? Je ne pense pas avoir la réponse absolue, cependant je pense pouvoir apporter au débat, un \textbf{recours citoyen} simple à mettre en oeuvre qui pourraient mener à de grandes améliorations dans le domaine de la gestion des risques:
\vspace{-8mm}

\begin{center}
\large{"Ouvrons nos données - C'est ce qu'on appelle l'\textit{Open Data} (cf. \nameref{ch:Lexique})"}
\end{center}

Ce n'est pas l'ouverture des données en soit qui permet de résoudre un quelconque problème, mais ce que de nombreux entrepreneurs décideront d'en faire très prochainement.

L'association OpenDataFrance peut justement être un support à ce genre d'initiatives.

Nous pouvons également cité une conférence qui a eu lieu lors du premier TEDxUTCompiègne, organisée en Janvier 2015 à l'UTC: \href{https://www.youtube.com/watch?v=MUI6Rwn4Qq0}{L’Open Data, Avenir des Big Data | Jean Marc LAZARD | TEDxUTCompiègne}.

Jean Marc LAZARD, PDG de OpenDataSoft y présente les enjeux de l'OpenData et ces usages.

\begin{figure}[H]
    \centering
    \includegraphics[height=13cm]{openDataFrance.png}
	\caption{Plaquette de Présentation de OpenDataFrance}\label{image.opendataFR}\textit{Source: \href{http://odfr.bype.org/wp-content/uploads/2012/02/Pr\%C3\%A9sentation-Opendata-France.pdf}{Lien vers la plaquette de présentation.}}
\end{figure}



\clearpage
