\section{"Concertation" ou le nouveau "\textit{Must-Have}".}

Fort de notre analyse précédente, il semble logique que la population et l'ensemble des acteurs qui travaillent à la limitation de l'impact des risques majeurs puisse échanger sur l'ensemble des sujets afin de relever dysfonctionnements et incohérences dans les divers processus mis en place. Ce serait également l'occasion de ne pas uniquement procéder à un flux ascendant de données mais également de sensibiliser le public à la question du management des risques majeurs qui pourraient les impacter plus ou moins directement et donc procéder à un échange montant et descendant d'informations.

C'est ce qu'on appelle la \textbf{Concertation}.

Soulignée comme la \textbf{"Solution"} avec un grand S, la concertation est mise en avant comme l'étape obligatoire qui résoudra dans un futur plus ou moins proche nos nombreux problèmes dans la gestion des risques majeurs naturels et technologiques.

Voici quelques citations extraites du corpus BC01 illustrant ce propos :

\begin{chapquote}
{Catherine CARRÉ et al\cite{Inondations}, \textit{page 7}}
L’opposition très forte des élus aux PPR\textbf{ est venue en partie de la dénonciation de l’absence de concertation} avec les services de l’État et du manque d’information de la société civile (Pottier et alii, 2004).
\end{chapquote}

\begin{chapquote}
{Catherine CARRÉ et al\cite{Inondations}, \textit{page 11}}
Vers \textbf{la reconnaissance de la concertation comme élément indispensable} de l’adaptation des territoires aux processus de gestion des risques.
\end{chapquote}

\begin{chapquote}
{Nicolas DECHY et al\cite{dechy}, \textit{page 8}}
Le Parlement européen a [...] demandé un changement de logique en raison de « l’impossible risque zéro » vers une logique « d’éloignement du risque ». Il a incité les Etats membres à revoir leurs politiques d’urbanisation autour des sites à risques \textbf{ainsi que les procédures de concertation avec le public}.
\end{chapquote}

\begin{chapquote}
{Laurent MICHEL\cite{LaurentMICHEL}, \textit{page 1}}
Après cette catastrophe, plusieurs enquêtes et de nombreux témoignages ont conduit le gouvernement à proposer une loi pour \textbf{renforcer la politique de prévention} des risques accidentels dans les installations classées, \textbf{sur quatre points qui en sont les piliers} [...] \textbf{3. La concertation et la participation du public et des salariés} [...]
\end{chapquote}

\begin{chapquote}
{Myriam MERAD\cite{merad}, \textit{Titre de l'article}}
« LA CONCERTATION : \textbf{UNE NOUVELLE DIMENSION} » 
\end{chapquote}

\subsection{Les \textbf{C}omités \textbf{L}ocaux d'\textbf{I}nformation et de \textbf{C}oncertation}

Les \textbf{C}omités \textbf{L}ocaux d'\textbf{I}nformation et de \textbf{C}oncertation, ou plus communément CLIC, ont été mis en place sur l'impulsion de la Loi 2003-699 du 30 Juillet 2003 \cite{PPRTech}, pour toute zone industrielle possédant au minimum une installation « Seveso avec servitude ». Ces Comités ont pour objectif d'encadrer et de rationaliser les échanges entre les acteurs et les populations directement concernées dans un objectif de prévention et ce durant la totalité de la durée de vie de ces installations à risque. Elles sont mises en place sous l'impulsion du préfet qui bénéficie d'un budget spécifique pour ces derniers. La mission des CLIC se résume donc en ces quelques mots :\\
\begin{itemize}
	\item Amélioration des échanges de l'information
	\item Impulser la collaboration entre les différents acteurs.
	\item Permettre une communication à destination du public.
	\item Organiser des débats sur les stratégies à adopter.
\end{itemize}

\section{Quelles sont les limites de la concertation ?}

\subsubsection{Avant-Propos}

Il est important de souligner que ce chapitre relève plus d'une \textbf{réflexion personnelle argumentée} que d'un fait avéré et prouvé mathématiquement. 

Il est tout à fait possible que mon argumentaire ne soit pas exact ou encore insuffisamment appuyé sur la littérature scientifique en de nombreux points. Cependant ce travail se positionne comme un\textbf{ travail d'écriture} bibliographique de fin d'études de cycle ingénieur. Il me semble donc primordial de ne pas rendre un travail passif de pure analyse et capitalisation de données mais bien d'inscrire celui-ci dans une logique ingénieur par \textbf{appropriation} du sujet et \textbf{remise en question} des postulats qui me semblent que partiellement exacts.


\subsection{Le parallèle Européen}

Afin d'illustrer mon propos je m'appuierai sur une situation actuelle que je pense similaire: L'exemple de la Démocratie Européenne. 

L'Europe a mis en place un système de concertation, où nous, citoyen européen, pouvons faire remonter un problème via un outils qui s'apparente aux pétitions appelé\textit{ Initiative citoyenne européenne} (\url{http://ec.europa.eu/citizens-initiative/public/welcome?lg=fr}). Si vous obtenez suffisamment de signatures (1 Million provenant d'au moins 7 Pays membres), votre requête sera entendue au parlement européen.

En théorie c'est fantastique, seulement quelles sont les limites pratiques d'un tel système (liste non exhaustive):\\
\begin{itemize}
	\item Qui peut se permettre de sacrifier des mois entiers pour obtenir les signatures nécessaires en plus de son travail régulier et de sa vie de famille ?\\
	
	\item Qui possède les connaissances linguistiques nécessaires pour s'adresser à sept pays différents ?\\
	
	\item Qui est même au courant de cette possibilité d'action ? \\
	
	\item Qui connait même le rôle de la Commission Européenne, qui sont nos élus ? Quels sont les sujets récents étudiés ?
	
\end{itemize}
 \vspace{5mm}

Ce n'est pas un malheureux hasard si le taux d'abstention lors de chaque élection européenne dépasse à chaque fois les 50\% (2014: 56\%, 2009: 59.36\%). Le système politique européen est obscure, difficile d'accès, et peu relayé par les médias.

Nous ne savons pas ce qui s'y fait et de quoi on y discute. A moins bien sûr de faire une démarche volontaire de recherche. Comportement que nous pouvons admettre, je pense, minoritaire.

\subsection{L'Europe, un modèle d'échec en terme de concertation}

Aujourd'hui, je considère l'Europe comme un \textbf{modèle d'échec} en terme de relations publiques, d'échanges avec les citoyens et de concertation. Les rares initiatives françaises de \textbf{référendum} se sont soldées par l'ignorance de l'avis du peuple (\textit{Non à la constitution Européenne}) et un passage en force suite à de légères modifications \textit{"pour la forme"} (Référendum du 29 mai 2005 en France).

Il me semble donc peu audacieux de conclure que volonté politique, enjeux économiques, et complexité d'un sujet mènent tous ensembles, non seulement au désintérêt des populations mais également à la désinformation.
Il est tellement plus facile d'influencer quelqu'un qui ne comprend qu'en apparence un sujet.

De plus, je ne crois pas qu'une population donnée de personnes concernées par un même risque ou problème soit dans la capacité d'apprécier le travail d'experts \textit{à l'état "brut"} et surtout de le remettre en question si nécessaire. Et dans ce cas on peut très justement se demander quel est l'intérêt d'obtenir un \textit{"Oui, je suis d'accord"} si ces personnes sont dans l'incapacité d'apporter un regard critique sur la situation.

\subsubsection{Quels sont les verrous d'une concertation efficace ?}

N'est-il pas, du moins en partie, hypocrite de rendre une ressource accessible en moyens et seulement profitable et exploitable par un comité réduit tout en ventant les mérites et les avancées pour tous que permettent cette accessibilité dédiée aux masses.

Il suffit de se rendre sur n'importe quel site ministériel, l'information est complexe à trouver si l'on ne sait pas où exactement chercher. Les sites sont très austères et ne sont en rien dédiés à la majorité. On pourrait facilement conjecturer que la complexité de ces ressources empêchent que le peuple se mêle trop de la vie politique mais là n'est pas le propos.

Fort de ces constats, voici une liste des critères qui sont essentiels, à mes yeux, pour assurer réussite d'un projet de concertation:\\
\begin{itemize}
	\item Une source d'information \textbf{officielle et centrale}. Il devrait être impensable de \textit{"jouer au ping pong"} entre les différents acteurs et services pour obtenir un quelconque renseignement.\\
	
	\item Un réel travail de \textbf{vulgarisation politico-technico-scientifique} doit être mené à tous les niveaux sur tous les supports de communication. Chaque personne concernée doit être capable de comprendre les enjeux et les leviers d'action qui s'offrent à elle. Nul ne devrait avoir besoin d'un Master en Management du Risque, en Chimie Industrielle, en Climatologie ou encore en Sciences Politiques pour participer aux débats\\
	
	\item Toujours dans un soucis d'accessibilité, il faut à tout prix éviter de noyer les comptes rendus de séances et autres documents officiels sous des montagnes de références législatives et scientifiques. Si ces rapports sont destinés à être partagés et donc à être lus, alors qu'ils soient \textbf{courts, clairs, simples et concis}. Un \textbf{travail de médiation} est à fournir. \\
	
	\item Dans la même lignée que le point précédent, l'architecture et le design des supports numériques \textit{(site-webs, applications mobiles, \ldots)} se doit d'être simple et attrayante. \textbf{Officiel et sérieux ne doit pas rimer avec austère et complexe}.\\	
	
	\item La communication doit-être totale et non en "Best Effort" \textit{(Anglais: Au mieux possible sans garantie)} comme pourraient dire les Américains. Pour toucher un large publique \textbf{aucun support ne doit être négligé}.\\
	
	\item Pour que \textbf{concertation} soit le mot adapté à la situation, il faut que les populations soient intégrées dès le lancement de chaque projet et non à la fin. Le cas contraire nous sommes dans une situation d'information qui cherche à donner l'illusion d'un semblant de contrôle. Il est évidement plus que souhaitable que ces populations possèdent un réel pouvoir décisionnel et non un rôle consultatif.\\
	
	\item Définir clairement les objectifs de chaque groupe de travail. Je ne crois pas que le fait de s'asseoir autour d'une table sans objectif et ordre du jour réglera les problématiques d'aujourd'hui. Chaque groupe de travail pouvant être focalisé sur des problématiques différentes ou même travailler sur des approches différentes d'une même problématique.\\
\end{itemize}
\vspace{5mm}


Par soucis d'honnêteté scientifique, je tiens cependant à souligner l'excellent travail, que j'ai eu la bonne surprise de découvrir, sur le site internet du gouvernement concernant les risques: \url{http://www.risques.gouv.fr/}. Cela montre une réelle volonté et motivation politique, essentielle à toute entreprise de ce type.

Cependant, ce site internet ne constitue qu'un seul des maillons élémentaires d'une longue chaine, et de très nombreux sont encore manquant pour mener une politique de concertation efficacement.

C'est un objectif qu'il faut évidement poursuivre mais qui n'est en rien, à mon humble avis, atteignable dans un futur proche.

\subsubsection{A quoi ressemblerait alors un recours citoyen envisageable dès demain ?}

Si la concertation est l'objectif de la semaine prochaine, alors nous sommes en droit de nous demander quel sera le recours citoyen de demain ? Il n'y a pas de réponse absolue à cette question, cependant je pense pouvoir apporter au débat, un \textbf{recours citoyen} simple à mettre en œuvre qui pourraient mener à de grandes innovations et améliorations dans le domaine de la gestion des risques naturels et technologiques:
\vspace{-8mm}

\begin{center}
\large{"Ouvrons nos données - C'est ce qu'on appelle l'\textit{Open Data} (cf. \nameref{ch:Lexique})"}
\end{center}

Ce n'est en rien l'ouverture des données publiques qui permet en soit de résoudre un quelconque problème, cependant cette ouverture permet dès lors une très grande variété d'études et d'applications. Le rôle de l'état n'est plus de pré-traiter, regrouper, trier, centraliser, dicter les usages et les objectifs. L'état et ses différentes instances \textit{se contentent} à présent uniquement de \textbf{publier l'ensemble de ces données} sur des plateformes libres et publiques. C'est maintenant aux acteurs privés, comme les assureurs, de considérer l'ensemble de ces données et d'en tirer les conclusions nécessaires. Il n'est plus nécessaire de capitaliser à l'extrême puisque l'ensemble des données se trouvent déjà en un point central.

La rationalisation des moyens et des structures permet également de créer des partenariats et opportunités que nuls n'aurait pu prévoir. La mouvance Open Data serait également un secteur extrêmement prolifique pour les StartUps Françaises qui  pourront analyser le sujet sur tous les angles, créer de nouveaux services et usages. L'OpenData permet la confrontation de \textbf{différents mondes d'expertises} qui n'auraient en d'autres lieux aucune raison de collaborer. Ces collaborations ne peuvent mener qu'à de nombreuses améliorations et créations d'emplois comme a pu le montrer l'exemple américain dans sa stratégie d'ouverture des données [Tim DAVIES et al.]~\cite{openData}.

L'association  \href{http://www.opendatafrance.net/}{OpenDataFrance} en complément de la mission gouvernementale \href{https://www.etalab.gouv.fr/}{ETALAB} peut justement être un support à ce genre d'initiatives.

\begin{figure}[H]
    \centering
    \includegraphics[height=12cm]{openDataFrance.png}
	\caption{Plaquette de Présentation de OpenDataFrance}\label{image.opendataFR}\textit{Source: \href{http://odfr.bype.org/wp-content/uploads/2012/02/Pr\%C3\%A9sentation-Opendata-France.pdf}{Lien vers la plaquette de présentation.}}
\end{figure}

Nous pouvons également cité une conférence qui a eu lieu lors du premier TEDxUTCompiègne, organisée en Janvier 2015 à l'UTC: \href{https://www.youtube.com/watch?v=MUI6Rwn4Qq0}{L’Open Data, Avenir des Big Data | Jean Marc LAZARD | TEDxUTCompiègne}. Le PDG de OpenDataSoft, une startup française spécialisée dans cette mouvance, y présente les enjeux de l'OpenData pour la France de demain et ces usages.





\clearpage
