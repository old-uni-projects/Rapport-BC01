\chapter*{Introduction}
\markboth{Introduction}{}
\addcontentsline{toc}{chapter}{Introduction}

Sécurité ! Un grand mot, une grande idée ! Cheval de bataille des démocraties modernes. Aujourd'hui il est nullement question des sujets très médiatiques comme la lutte anti-terrorisme. Aujourd'hui, nous nous attarderons sur les \textbf{risques technologiques et naturels}. Étant Toulousain, la journée du 21 Septembre 2001, jour de l'explosion de l'usine AZF, est encore bien présente dans ma mémoire. Mais avant d'aborder ce triste souvenir, revenons à l'\textit{état de l'art} de cette discipline.

L'\textbf{idée de risque} est en apparence simple et pourtant complexe à définir (cf. \nameref{ch:Lexique}). Cette dernière peut-être, par exemple, encadrée par la notion mathématique : $ Risque = Aléa * Vulnérabilité $ (cf. \nameref{ch:Lexique}). Mais également comme un évènement à caractère négatif (\textit{inondation, \ldots}) aux conséquences néfastes. Nous nous en tiendrons à ces deux visions principales. 

\textit{Pour la petite anecdote, nous pouvons situer, assez précisément dans le temps, le premier accident technologique. Ce dernier remonte, assez logiquement, aux débuts de la \textbf{maitrise du feu}, qui comme on peut s'en douter  facilement, a dû mener à quelques incendies involontaires du fait de la fraiche maitrise du phénomène. Risque qui est, d'ailleurs, toujours d'actualité \ldots}

Tout citoyen peut aujourd'hui s'informer, s'il le désire, sur les risques majeurs auxquels il pourrait être amené à faire face à plus ou moins courte échéance. Il peut également prendre connaissance des études prévisionnelles concernant les dommages prévisibles et les mesures misent en place pour les minimiser.\\
\textbf{La communication, à l'attention de la population}, autour de ces derniers \textbf{se pose comme un élément essentiel} de la prévention des risques majeurs naturels et technologiques.

Et pourtant cela n'a pas toujours été une évidence, c'est seulement depuis la loi du 22 Juillet 1987~[Texte de la Loi : \url{http://www.ineris.fr/aida/consultation_document/2183}] que le droit à l'information, pour les populations, concernant les risques majeurs naturels et technologiques, est inscrit dans la loi.
Les articles L.125-2~\cite{L125-2}, L.125-5~\cite{L125-5} et L.563-3~\cite{L563-3} du Code de l'environnement viennent renforcer et encadrer ce droit. 

%Comment ces articles définissent les grandes lignes de la prévention des risques ? Quelles sont leur conséquences et leur impacts ? Ces dernières ont-elles \textit{effacé} le sentiment de danger ou d'insécurité ? Apportent-elles une réponse cohérente au besoin ? Cette réponse est-elle suffisante ? Qui est chargé de la vérification de leur application ? Comment pourrions-nous les améliorer ? Quelles sont les limites d'un système législatif concernant les contrôles du risque ?

Après une courte analyse de la situation actuelle en France au regard de la gestion des risques naturels ou technologiques, à la fois d'un point de vue humain, environnemental et financier. Nous étudierons une phénomène montant: \textbf{La Concertation entre tous les acteurs et la population}. Nous nous demanderons, comment et pourquoi se dernier se place comme l'atout indispensable d'une politique de gestion des risques? \\
Enfin dans un dernier temps nous analyserons les limites pratiques d'une telle politique qui est non seulement à la mode mais se pose comme l'\textbf{unique solution} à une gestion efficace des risques. Nous verrons que si la concertation se pose comme un objectif long terme, la mouvance \textit{Open Data} (cf. \nameref{ch:Lexique}) se place comme le catalyseur de l'innovation sociale dans la gestion des risques majeurs.

Pour répondre à l'ensemble de ces questions je m'appuierai sur le corpus fourni dans le cadre de cette étude ainsi que plusieurs références externes à ce dernier. La liste complète de ces documents figure dans la bibliographie de ce rapport.

%\clearpage