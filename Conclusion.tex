\chapter{Conclusion}

C'est donc avec espoir que nous pouvons aujourd'hui regarder la gestion des risques majeurs naturels et technologiques en France. Il semblerait que l'Humain et sa société dans son ensemble aient été replacés au centre du processus décisionnel. 

Grâce à l'étude des \textbf{aléas} d'un point de vue scientifique, nous avons pu nous munir d'un ensemble de détecteurs qui nous permettent de réagir au plus vite en situation de crise.

Grâce à la mise en place de \textbf{plans de prévention}, nous avons pu réduire la vulnérabilité économique et humaine dans nos sociétés.

Grâce à une \textbf{coordination des différents acteurs}, nous avons pu établir des plans d'actions pour gérer la crise quand elle se présente.

De nos jours, il semblerait que la concertation citoyenne se positionne positivement et progresse à grands pas pour révolutionner la gestions des risques majeurs naturels et technologiques. La mouvance Open Data sera peut-être et même certainement le tremplin pour propulser cette dernière au rang des politiques à succès et diminuer le risque dans nos sociétés.

Bien qu'il reste évidement de nombreux points d'amélioration, je pense que la position politique proactive de la France concernant de la gestion des risques majeurs naturels et technologiques pourra, sur le long terme, non seulement réduire le risque et améliorer sa gestion mais également créer de l'innovation, de l'expertise et de l'emploie dans nos régions.