\chapter*{Lexique}
\markboth{Lexique}{}
\label{ch:Lexique}
\addcontentsline{toc}{chapter}{Lexique}

\begin{Define}{Le Risque}{}
De nos jours, le risque se pose comme l'«éventualité d'un événement futur, incertain ou d'un terme indéterminé, ne dépendant pas exclusivement de la volonté des parties et pouvant causer la perte d'un objet ou tout autre dommage (Cap. 1936) ».

[\url{http://www.cnrtl.fr/definition/risque}]
\end{Define}

\begin{Define}{Un Risque Majeur}{}
« La possibilité d’un événement d’origine naturelle ou humaine, dont les effets peuvent menacer la population, occasionner des dommages importants. Le risque majeur est caractérisé par sa faible fréquence et son énorme gravité. » [\url{http://www.risques.gouv.fr/}]
\end{Define}

\begin{Define}{Un Aléa}{}
«Risque, inconvénient que l'on envisage sans pouvoir l'imaginer avec précision ou le situer avec exactitude dans le temps ». [\url{http://www.cnrtl.fr/definition/alea}]. Un aléa devient dangereux et constitue donc un risque si et seulement si ce dernier peut avoir un effet regrettable.\\
\textbf{Un \textit{Risque} est donc un \textit{Aléa}, mais la réciproque n'est pas systématiquement vrai.}
\end{Define}

\begin{Define}{Un risque naturel majeur}{}
Un risque naturel majeur est une menace ayant pour origine des phénomènes naturels \textit{(exemples: géologiques, atmosphériques, ...)} non prévisibles. Ces derniers provoquent des dégâts importants sur les populations, leurs biens, et leurs environnements.
\end{Define}

\begin{Define}{Un risque technologique majeur}{}
Un risque technologique majeur est une menace ayant pour origine l'activité humaine. C'est un événement non souhaité engendré par le dysfonctionnement accidentel d'un système ou processus qui est potentiellement dangereux pour la population, l'environnement, etc \ldots\\
Ses conséquences peuvent être plus ou moins graves, immédiates ou inscrites dans le temps, avec une portée géographique plus ou moins étendue.
\end{Define}

\begin{Define}{Open Data}{}
L’ouverture et le partage des données publiques, aussi appelés Open Data, consistent à mettre à disposition de tous les citoyens, sur Internet, toutes les données publiques brutes qui ont vocation à être librement accessibles et gratuitement réutilisables [Etalab, 2013].
\end{Define}

\begin{Define}{Les types de risques}{}
Le ministère de l’écologie et du développement durable a identifié :
\vspace{2mm}
\begin{itemize}
	\item \textbf{Onze risques naturels majeurs:} 
		\begin{itemize}
			\item Avalanche
			\item Canicule
			\item Cyclone
			\item Éruption Volcanique
			\item Feux de forêts
			\item Grand Froid
			\item Inondation
			\item Mouvement de Terrain
			\item Séisme
			\item Tempête
			\item Tsunami
		\end{itemize}	
	\vspace{2mm}
	\item \textbf{Cinq risques technologiques majeurs:} 
		\begin{itemize}
			\item Accident industriel
			\item Accident nucléaire
			\item Risque Minier
			\item Rupture de barrage
			\item Transport de Matières Dangereuses
		\end{itemize}	
\end{itemize}
\vspace{2mm}
Source: \url{http://www.risques.gouv.fr/}
\end{Define} 
\vspace{-3mm}

\begin{Define}{Degré de Vulnérabilité}{}
Un risque de même nature et de même intensité ne provoque pas forcément les mêmes dégâts en différentes localisations.\\

Cela est du à plusieurs facteurs :
\vspace{2mm}
\begin{itemize}
	\item \textbf{Des facteurs techniques:} Qualité des constructions, structures préventives, ...
	\item \textbf{Des facteurs économiques:} Les populations moins aisées ont plus de mal à fuir le danger, se procurer un moyen de transport, ...
	\item \textbf{Des facteurs administratifs et politiques:} La présence ou l'absence de coordination entre les différents acteurs pour prévenir le risque ou gérer la crise, le niveau d'information de la population, ...
\end{itemize}
\end{Define}
\vspace{-3mm}

\begin{figure}[H]
    \centering
    \includegraphics[height=4.25cm]{defineRecap.jpg}
	\caption{Aléa, Vulnérabilité, Risque et Catastrophe/Crise}\label{image.defineRecap}\textit{Source: \url{http://www.saintmaurautrement.com/?p=908}}
\end{figure}

\clearpage